
\begin{tabularx}{\linewidth}{@{}c@{}cX}
&\#Studies & Summary\\
\cite{Peng2013UsingResearch}& 28 & Exertion games might help inactive children to start exercising, but aren't a substitute for exercising because only light to moderate exercise. \\
\cite{Chaput2013ActiveYouth}& 52 & Don't recommend exertion games as intervention for children because intensity isn't high enough. \\
\cite{LeBlanc2013ActiveReview}& 21 & All studies reported energy expenditure above rest, but sometimes below comparison exercises. \\
\cite{biddiss2010active}&  12 & Exertion games are light-moderate activity, activity levels highly variable, lower body games higher than upper body. \\
\cite{Sween2014TheReview}& 27 & The majority of studies exhibited at least moderate exercise.\\ 
\cite{liang2014effects}& 32 & Exertion games were light to moderate exercise.\\
\cite{daley2009can}& 14 & Exertion games do promote exercise, but aren't as intense as the activities they simulate.\\
\cite{gao2015meta}& 35& In the lab, games are better than a slow walk, but worse than 5.7km/h walking or running, but against field studies exercise results are mixed.\\
\cite{Lee2017TheReview}& 45 & Enjoyment of exertion games predicts exercise levels. \\
\cite{Hchsmann2016EffectsIndividuals}& 11& Overweight gamers don't achieve moderate intensity exercise.\\
\cite{Lamboglia2013ExergamingReview}& 9 & Exertion gaming increases heart rate and other physiological measures.\\
\cite{Whitehead2010ExergameUs} & 15 & Wide range of energy expenditure in different games.\\
\cite{Foley2010UseReality} &11& Exertion games are mild to moderate exercise. \\
\cite{Parisod2014PromotingReviews}& 12 & Review of reviews. Exertion games have positive effects on energy expenditure, but magnitudes are variable. \\

\end{tabularx}
%\cite{Peng2013UsingResearch} - systematic review of 28 lab studies, shows that they increase activity over rest. Recommends: might be useful for encouraging otherwise inactive children to start exercising, but aren't a substitute for exercising in children because only light to moderate exercise.

%\cite{Chaput2013ActiveYouth} systematic Review by Active Healthy Kids Canada, to consider recommendations in relation to exertion games for children. Don't recommend them as intervention for children because the intensity isn't high enough, but suggest in special populations and rehab they might be useful.

%\cite{LeBlanc2013ActiveReview} Reviewed 21 studies relating to acute energy use of exertion games in children, again noting that all studies reported levels of EE above rest, although sometimes below comparison exercises.

%\cite{biddiss2010active} 12 studies - AVGs enable light-moderate activity, activity levels highly variable, lower body games higher than upper body. 

%\cite{Sween2014TheReview} Reviewed 27 studies, says that the majority of studies exhibited at least moderate exercise 

%\cite{liang2014effects} children - 32 studies - exergames light to moderate exercise.

%\cite{daley2009can} Reviews 14 studies, notes that they increase exercise, but that they aren't as intense as the activities they pretend to be.

%\cite{gao2015meta} 35 articles - AVGs vs lab based exercise - AVGs better than slow walk (2.6km/h) and (4.2km/h), equal to biking but worse that 5.7km/h walking or running. AVGs vs field studies negligible difference between objective measures of exercise vs AVGs.

%\cite{Lee2017TheReview} reviews psychological effects of exergame play, and relates them to physiological effects (i.e. enjoyment -> EE)

%\cite{Hchsmann2016EffectsIndividuals} overweight -11 studies much variation, most studies didn't go >3 METs.

%\cite{Lamboglia2013ExergamingReview} 9 articles - exergaming increases HR etc.

%\cite{Whitehead2010ExergameUs} reviews 15 studies - shows wide range of energy expenditure, although units are unclear

%\cite{Foley2010UseReality} 11 studies - mild to moderate exercise

