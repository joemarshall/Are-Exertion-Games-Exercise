\section{Introduction}
% The very first letter is a 2 line initial drop letter followed
\IEEEPARstart{E}{xertion} games are games in which the core play activities involve the player engaging in exercise. Academic interest in such games began when researchers observed the amount of exercise done by players of the arcade game Dance Dance Revolution \cite{tan2002aerobic}. Both academics and commercial game developers have since built a wide range of games deliberately designed to encourage participants to exercise, with some, such as the Wii Fit, being extremely commercially successful.

Exertion games are often presented as health interventions, for example as solutions to increasing childhood sedentary behaviour and obesity \cite{marshall2017misrepresentation}. For these games to have positive effects on health via exercise, 3 things must be true \cite{Dutta2015EffectsReview}:
\begin{enumerate}
    \item Exertion games must induce sufficient exercise levels in the short term to have positive health effects.
    \item People must choose or be encouraged to play over the long time periods required to achieve health effects.
    \item Opportunity costs of exertion games must not outweigh any direct health benefits. For example, there is concern that money spent on games may replace exercise equipment, or time spent playing would replace time spent doing sports activities with proven health benefits \cite{Dutta2015EffectsReview}.
\end{enumerate}
In this article we produce a scoping literature review \cite{arksey2005scoping} of empirical studies addressing the first of these questions. 

\section{Background}

\subsection{What are exertion games and why are people excited?}

We define exertion games as games that require energetic movement to play. Examples of such games include fitness focused games that deliberately aim to induce exercise, such as the Wii Fit series (Nintendo, 2007-2014), and Xbox Fitness (Microsoft, 2013-201), along with full body movement games not framed as 'exercise games', such as Dance Dance Revolution (Konami, 1998-2016), Wii Sports (Nintendo, 2006-2018), Kinect Adventures / Sports (Microsoft, 2010-2014).  

A wide range of research papers have put forward exertion games as solutions to identified health problems, most commonly as solutions to perceived low levels of physical activity in children, and increased obesity levels in many populations \cite{marshall2017misrepresentation}. Such research often takes the view that people play games frequently, and that choosing to play exertion games instead of sedentary games would lead to health benefits. The specialist conferences and journal of 'Games for Health' have included a wide range of exertion gaming research.

\subsection{What do existing reviews say?}

We are not the first to review evidence for exertion games as health interventions. We summarise here existing reviews relating to short term exercise effects of exertion games. 

Firstly, we found 3 meta reviews. The first considered 18 studies and found exertion games caused higher heart rates and oxygen consumption than rest \cite{Peng_2011}; and that exertion games were similar in exertion level to exercise controls. Lower body exertion games were more energetic than upper body games. Two other studies considered metabolic equivalent of task measures of exertion game exercise levels, one finding an average level of 2.6 METs (light exercise) from 15 studies \cite{Dutta2015EffectsReview}, and a second finding 3.1-3.2 METs from 9 studies \cite{barnett2011active}, within the moderate level of exercise recommended by public health organisations. We also found a range of systematic reviews, which considered from 9 to 52 studies in total and had variable results (see  Table \ref{tab:reviewofreviews} for a summary of these studies).   

\begin{table}[]
    \caption{Existing reviews of exertion games}
    \label{tab:reviewofreviews}

\begin{tabularx}{\linewidth}{@{}c@{}cX}
&\#Studies & Summary\\
\cite{Peng2013UsingResearch}& 28 & Exertion games might help inactive children to start exercising, but aren't a substitute for exercising because only light to moderate exercise. \\
\cite{Chaput2013ActiveYouth}& 52 & Don't recommend exertion games as intervention for children because intensity isn't high enough. \\
\cite{LeBlanc2013ActiveReview}& 21 & All studies reported energy expenditure above rest, but sometimes below comparison exercises. \\
\cite{biddiss2010active}&  12 & Exertion games are light-moderate activity, activity levels highly variable, lower body games higher than upper body. \\
\cite{Sween2014TheReview}& 27 & The majority of studies exhibited at least moderate exercise.\\ 
\cite{liang2014effects}& 32 & Exertion games were light to moderate exercise.\\
\cite{daley2009can}& 14 & Exertion games do promote exercise, but aren't as intense as the activities they simulate.\\
\cite{gao2015meta}& 35& In the lab, games are better than a slow walk, but worse than 5.7km/h walking or running, but against field studies exercise results are mixed.\\
\cite{Lee2017TheReview}& 45 & Enjoyment of exertion games predicts exercise levels. \\
\cite{Hchsmann2016EffectsIndividuals}& 11& Overweight gamers don't achieve moderate intensity exercise.\\
\cite{Lamboglia2013ExergamingReview}& 9 & Exertion gaming increases heart rate and other physiological measures.\\
\cite{Whitehead2010ExergameUs} & 15 & Wide range of energy expenditure in different games.\\
\cite{Foley2010UseReality} &11& Exertion games are mild to moderate exercise. \\
\cite{Parisod2014PromotingReviews}& 12 & Review of reviews. Exertion games have positive effects on energy expenditure, but magnitudes are variable. \\

\end{tabularx}
%\cite{Peng2013UsingResearch} - systematic review of 28 lab studies, shows that they increase activity over rest. Recommends: might be useful for encouraging otherwise inactive children to start exercising, but aren't a substitute for exercising in children because only light to moderate exercise.

%\cite{Chaput2013ActiveYouth} systematic Review by Active Healthy Kids Canada, to consider recommendations in relation to exertion games for children. Don't recommend them as intervention for children because the intensity isn't high enough, but suggest in special populations and rehab they might be useful.

%\cite{LeBlanc2013ActiveReview} Reviewed 21 studies relating to acute energy use of exertion games in children, again noting that all studies reported levels of EE above rest, although sometimes below comparison exercises.

%\cite{biddiss2010active} 12 studies - AVGs enable light-moderate activity, activity levels highly variable, lower body games higher than upper body. 

%\cite{Sween2014TheReview} Reviewed 27 studies, says that the majority of studies exhibited at least moderate exercise 

%\cite{liang2014effects} children - 32 studies - exergames light to moderate exercise.

%\cite{daley2009can} Reviews 14 studies, notes that they increase exercise, but that they aren't as intense as the activities they pretend to be.

%\cite{gao2015meta} 35 articles - AVGs vs lab based exercise - AVGs better than slow walk (2.6km/h) and (4.2km/h), equal to biking but worse that 5.7km/h walking or running. AVGs vs field studies negligible difference between objective measures of exercise vs AVGs.

%\cite{Lee2017TheReview} reviews psychological effects of exergame play, and relates them to physiological effects (i.e. enjoyment -> EE)

%\cite{Hchsmann2016EffectsIndividuals} overweight -11 studies much variation, most studies didn't go >3 METs.

%\cite{Lamboglia2013ExergamingReview} 9 articles - exergaming increases HR etc.

%\cite{Whitehead2010ExergameUs} reviews 15 studies - shows wide range of energy expenditure, although units are unclear

%\cite{Foley2010UseReality} 11 studies - mild to moderate exercise


\end{table}

\begin{figure}[]

\tikzset{
    flownode/.style={
        draw, rectangle, align=center, text width=5cm, font=\small, inner sep=3ex},
    splitnode/.style={
        draw, rectangle, align=center, text width=1.25cm,minimum height=1.5cm, font=\small, inner sep=1ex},        
    flowlabel/.style={
        draw, rectangle, align=center, rounded corners, font=\small\bf, inner sep=2ex, 
        fill=cyan!30, minimum height=3.8cm},
    arrow/.style={
        very thick,->,>=stealth}
}
\centering
\begin{tikzpicture}[
    node distance=.5cm,
    start chain=1 going below,
    every join/.style=arrow,
    ]
    % the chain in the center going below
    \coordinate[on chain=1] (tc);
    
    \node[flownode,text width=8cm,align=left, inner sep=1ex,on chain=1](n0)
    {
\textbf{Scholar title searches:}
\textit{exertion game, exercise game, "active video game", exergame, health game, game "for health", "games for health", fitness game.}

\textbf{Pubmed:} \textit{exergame OR (exertion AND game) OR ((active OR exercise OR exertion OR fitness) AND ("video game" OR "computer game" OR "digital game") )}

\textbf{Scopus:} \textit{TITLE-ABS-KEY ( exergame) OR TITLE-ABS-KEY ("exertion game") OR ((TITLE-ABS-KEY (active) OR TITLE-ABS-KEY (exercise) OR TITLE-ABS-KEY (exertion) OR TITLE-ABS-KEY (fitness) ) AND (TITLE-ABS-KEY ("video game") OR TITLE-ABS-KEY ("computer game") OR TITLE-ABS-KEY ("digital game")) )}
    };
    \node[flownode, join,on chain=1](n1)
        {3171 records from searches};
.    \node[flownode, join, on chain=1] (n3)
        {2665 non-duplicate records screened};
    \node[flownode, join, on chain=1] (n5)
        {726 exertion game articles in study};
    % the five individual research questions
  \node[splitnode, below=1cm of n5, anchor=north] (rq3)
        { 76  long term health studies };
    
   \node[splitnode, left=.1cm of rq3] (rq2)
      { 133  long term exercise studies };
    
   \node[splitnode, left=.1cm of rq2] (rq1)
        { 243 short term exercise studies };



  \node[splitnode, right=.1 cm of rq3] (rq4)
        { 37  cognitive effects studies};

  \node[splitnode, right=.1cm of rq4] (rq5)
        { 291  balance and rehab };

    \draw[arrow] (n5) -- (rq1.north);
    \draw[arrow] (n5) -- (rq2);
    \draw[arrow] (n5) -- (rq3);
    \draw[arrow] (n5) -- (rq4);
    \draw[arrow] (n5) -- (rq5.north);
    
\end{tikzpicture}
\caption{Search process and numbers of articles found} \label{fig:flowchart}
\end{figure}



%About measurement of things through accelerometry
%\cite{Tripette2014EvaluationMeasurements}

%About exertion games in rehab for elderly (review) generally good at balance improvement, offer some MVPA
%\cite{Ruivo2014ExergamesRehabilitation}

%About how exertion games might aim for movement diversity
%\cite{Landry2013DesignChildren}

\subsection{How do we go beyond those existing meta reviews?}

Existing reviews considering the question of whether games are useful exercise have reached widely differing conclusions, from \enquote{light-to-moderate intensity of AVG play does not meet the daily 60-minute MVPA goal for children.\cite{Peng2013UsingResearch}} to \enquote{exergaming can increase EE to levels that meet the ASCM recommended guidelines for health and fitness} \cite{Sween2014TheReview}. 

The reason for these disparities is the complexity of the question being asked. In fact, we would argue that the answer to the question 'are games exercise?' is so clearly contingent on the nature of the game being played and the level it is played at, that one cannot review exergaming as a single whole. Combining results in meta reviews, to come to conclusions such as 'overall metabolic equivalent associated with AVG play was 2.62'\cite{Dutta2015EffectsReview}, is lacking in meaning when, as we see in our review, this can mask a range from 2 to 9 METs. We also believe that because games are diverse, a review which uses highly rigorous selection criteria will inevitably lead to a small sample of the possible research on exertion games, and a different sample to those in other reviews - indeed in Parisod et al's 'review of reviews' in this area\cite{Parisod2014PromotingReviews}, we see that the vast majority of studies included in any of the 15 reviews they consider are only included in 1 review, with each study considering a completely different subset of the literature. Instead, we deliberately use highly inclusive selection criteria, to present a broader picture of the wide range of existing research results relating to these games, and do not simply aggregate what is clearly a highly heterogenous set of results. 

\section{Data gathering}

We ran an initial pilot using google scholar title searches, and used this to identify 5 key categories of research relating to exertion games and health which are shown at the bottom of Figure \ref{fig:flowchart}. Of these, this paper focuses only on the first category, studies of the acute, short term effects of exertion games. We then did full searches in scopus, pubmed and google scholar, screened for duplicates. In a paired analysis process with 2 raters, we identified 726 exertion game articles, and filtered these into the 5 categories (see Figure \ref{fig:flowchart}). Of these, 243 studies considered short term exercise effects of exertion games.  


\section{Results}

We split the analysis of short term effects of exertion games into discussion of 4 questions:
\begin{itemize}
    \item How much exercise are exertion games (i.e. quantitative measurements of exercise intensity)?
    \item How do exertion games compare to other forms of exertion (both in terms of exercise amount and enjoyment, immersion, perceived level of exertion etc.)
    \item How do different features of exertion games alter participant exercise levels and experience?
    \item How do different user groups experience exertion games differently?
\end{itemize}



\subsection{How Intense Exercise are Exertion Games?}

We considered all studies which took an absolute measure of exercise intensity in one or more identifiable exertion games. We extracted the mean intensity of exertion and converted to Metabolic Equivalent of Task ( $1~MET= \frac{1~kcal}{ kg \cdot h} = \frac{3.5~ml~O_2}{ kg \cdot h}  $). For some  articles, this was not possible, as they used energy expenditure or VO2 without normalising for weight (e.g.\cite{Gribbon2015ActiveTrial}), or measures such as minutes above an HR threshold \cite{MacIntosh2017BalancingIii} or activity counts \cite{McDougall2008ChildrenStudy}. In Figure \ref{tab:METs}, we plot 207 data points from 48 articles, each showing a measurement of mean MET for an individual game. Of these points, 94 measurements were below 3 METs, so did not achieve moderate levels of exercise. 92 achieved moderate activity ( 3-6 METs), and 20 achieved vigorous activity (\textgreater 6 METs).
  
\begin{table}
\setlength{\abovecaptionskip}{0pt}
\vspace{-4mm}
\caption{Graph of metabolic rates observed in exertion game studies.}
\pgfplotstableread[col sep=comma]{met-analysis-latex.csv}\datatable

\newcommand{\METLine}[1]{\addplot[only marks,mark options={fill=black,fill opacity=.8,draw opacity=0}] table [y expr=\coordindex
,x=#1]{\datatable};
}

\begin{tikzpicture}
\begin{axis}[small,height=24cm,width=6cm,xmin=0,
xmax=10,
enlarge y limits=.01,
yticklabels from table={\datatable}{Game Cited},
y tick label style={font=\tiny,text width=4cm,align=right},
ytick = data,
xtick={0,...,10},
y dir = reverse,
xlabel={Mean level of exertion (METs)}
]
\addplot[xbar,fill=lightgray,fill opacity=.2,draw=none] table[y expr=\coordindex,x expr=\thisrow{alternating}*10]{\datatable};
\filldraw[fill=yellow,fill opacity=.1,draw=none] ({axis cs:6,0}|-{rel axis cs:0,1}) rectangle ({axis cs:10,0}|-{rel axis cs:1,0});
\filldraw[fill=green,fill opacity=.1,draw=none] ({axis cs:3,0}|-{rel axis cs:0,1}) rectangle ({axis cs:6,0}|-{rel axis cs:1,0});

\METLine{MET1};
\METLine{MET2};
\METLine{MET3};
\METLine{MET4};
\METLine{MET5};
\METLine{MET6};
\METLine{MET7};
\METLine{MET8};
\METLine{MET9};
\METLine{MET10};
\METLine{MET11};
\METLine{MET12};
\METLine{MET13};
\METLine{MET14};
\METLine{MET15};
\METLine{MET16};
\METLine{MET17};
\METLine{MET18};
\METLine{MET19};
\METLine{MET20};
\METLine{MET21};
\METLine{MET22};
\METLine{MET23};
\METLine{MET24};
\METLine{MET25};
\METLine{MET26};
%\addplot[only marks] table [y expr=\coordindex,x=MET1]{\datatable};
\end{axis}
%\filldraw [brown] (current bounding box.north west) rectangle (current bounding box.south east);
\node[draw] at (0,-1.2) {Green bar = moderate exercise (3-6 METs), yellow bar = vigorous exercise (6+ METs)};
 

\end{tikzpicture}
\label{tab:METs}
\end{table}

World Health Organisation \cite{world2010global}, USA \cite{usguidelines2018} and \cite{ukguidelines2011} British recommendations for adults all prescribe at least 150 minutes of moderate physical exercise (3-6 METs) per week or 75 minutes of vigorous exercise (6+ METs). For children, they also prescribe 3 sessions/week of vigorous activity (e.g. jogging). Only 20 measurements (9.8\% of results) showed vigorous exercise. Further, whilst exertion games may contribute to fulfilling recommendations through moderate exercise, in almost 45\% of studies, people did not even reach moderate levels. If people in optimum conditions of academic health labs, doing studies about exercise, whilst wearing exercise measurement equipment, do not exercise hard enough, it is not strong support for use of exertion games as a health intervention, as others have noted, for example Scheer et al. conclude that \enquote{Overall, playing a “physically active” home video game system does not meet the minimal threshold for moderate intensity physical activity, regardless of gaming system\cite{scheer2014wii}}. Strong results for some specific games suggest that extreme variance in results is dependent on game design, for example dance games almost all achieve moderate exercise in our results, and as Fawkner et al. \cite{Fawkner2010AdolescentGaming} note, may offer real potential for exercise particularly in specific at risk groups such as adolescent girls. 

The wide ranges of measurements for popular games, highlight that exertion may vary strongly based on chosen difficulty levels, skill or other factors. This shows that exertion games can be effective exercise if played in the right way, but that how the game is played can vastly alter intensity of exercise.

%Whilst there is increasing amounts of evidence that purely anti-sedentary behaviour such as taking breaks from sitting can have positive health benefits, if we are to make games which specifically encourage exercise, rather than just to encourage 

\subsection{How do games compare to other forms of exercise?}

It is clear that exertion games have the potential to be exercise. A second question commonly addressed by research is how this exercise compares to existing forms of exercise. This work explores whether games have the potential to make exercise more enjoyable, or to encourage people to exercise harder, perhaps by distracting them from how hard they are working. Results are shown in tables \ref{tab:gamevsexercise} and \ref{tab:enjoyment}.

\begin{table}
\caption{Measured Activity Level in Exertion Games versus Traditional Exercise Activities}
\label{tab:gamevsexercise}
\centering

{
\setlength{\tabcolsep}{1pt}
\begin{tabular}{lrcl}
\hline
& \multicolumn{1}{r}{\textbf{Exertion Game}}
 & & \multicolumn{1}{l}{\textbf{Exercise Activity}}
 \\
\hline

% table main starts here
\cite{Bailey2011EnergyExergaming}&Cybex Trazer&\cellcolor{green!30}$>$&Treadmill at 4.8 km/h\\
\cite{Gao2013ChildrenSDance}&DDR&\cellcolor{red!30}$<$&Aerobic Dance\\
\cite{Kraft2011HeartParticipants}&DDR&\cellcolor{red!30}$<$&Gamebike, self selected pace\\
\cite{Kraft2011HeartParticipants}&DDR&\cellcolor{yellow!30}$=$&Stationary Bike + Tv, self selected pace\\
\cite{Bailey2011EnergyExergaming}&DDR&\cellcolor{yellow!30}$=$&Treadmill at 4.8 km/h\\
\cite{LanninghamFoster2006EnergyChildren}&DDR&\cellcolor{yellow!30}$=$&Treadmill + Tv at 2.4 km/h\\
\cite{Graf2009PlayingChildren}&DDR&\cellcolor{yellow!30}$=$&Walking at 5.7 km/h\\
\cite{Haddock_2009}&Disney Cars&\cellcolor{green!30}$>$&Stationary Bike\\
\cite{delCorral2014PhysiologicalCf}&EA Sports&\cellcolor{green!30}$>$&6 Minute Walk Test, self selected pace\\
\cite{Legear2016DoesCopd}&EA Sports&\cellcolor{yellow!30}$=$&Treadmill at "moderate effort"\\
\cite{Perron2012ComparisonAdults}&EA Sports&\cellcolor{black!20}Mixed&Treadmill at "brisk walk"\\
\cite{Glen2017ExergamingHarder}&Expresso HD bike&\cellcolor{green!30}$>$&Just Exercise Bike, self selected pace\\
\cite{Matrosly2017ExergamingInjury}&Eyetoy&\cellcolor{yellow!30}$=$&Heavy Bag Boxing\\
\cite{LanninghamFoster2006EnergyChildren}&Eyetoy&\cellcolor{yellow!30}$=$&Treadmill + Tv at 2.4 km/h\\
\cite{delCorral2014PhysiologicalCf}&Family Trainer&\cellcolor{green!30}$>$&6 Minute Walk Test\\
\cite{Pasco2017TheMotivation}&Gamebike&\cellcolor{red!30}$<$&Stationary Bike, self selected pace\\
\cite{Monedero2015InteractiveAdults}&Gamebike&\cellcolor{green!30}$>$&Stationary Bike at 55\% peak power\\
\cite{Warburton2009MetabolicCycling}&GameBike&\cellcolor{green!30}$>$&Stationary Bike at 25-50\% peak power\\
\cite{Kraft2011HeartParticipants}&Gamebike&\cellcolor{green!30}$>$&Stationary Bike + Tv, self selected pace\\
\cite{Fitzgerald2004TheErgometry}&GameCycle&\cellcolor{green!30}$>$&Arm Ergometer\\
\cite{OConnor2002KineticSystem}&GameWheels&\cellcolor{green!30}$>$&Wheelchair Treadmill\\
\cite{Rincker2017TheFitness}&Irish dance game&\cellcolor{red!30}$<$&Irish Dance Lessons\\
\cite{Lin2015JustOutcomes}&Just Dance&\cellcolor{yellow!30}$=$&Dancing To Online Videos\\
\cite{eason2016comparison}&Just Dance&\cellcolor{red!30}$<$&Dance Fitness Class\\
\cite{Vallabhajosula2016EffectStudy}&Kinect Adventures&\cellcolor{yellow!30}$=$&School Recess (Free Play)\\
\cite{Lisn2015CompetitiveAdolescents}&Kinect Adventures&\cellcolor{green!30}$>$&Treadmill at 4.2 km/h\\
\cite{Monedero2017EnergyExercise}&Kinect Adventures&\cellcolor{red!30}$<$&Treadmill Running , self selected pace\\
\cite{Monedero2017EnergyExercise}&Kinect Adventures&\cellcolor{red!30}$<$&Treadmill Running  at 55\% VO2 reserve\\
\cite{Bailey2011EnergyExergaming}&Light Space&\cellcolor{green!30}$>$&Treadmill at 4.8 km/h\\
\cite{Chittaro2012TurningWalking}&LocoSnake&\cellcolor{yellow!30}$=$&Walking around a field\\
\cite{Sun2012ExergamingChildren}&Mixed exergaming&\cellcolor{red!30}$<$&Physical Education Lesson\\
\cite{Fogel2010TheClassroom}&Mixed exergaming&\cellcolor{green!30}$>$&Physical Education Lesson\\
\cite{Shayne2012TheClass}&Mixed exergaming&\cellcolor{green!30}$>$&Physical Education Lesson\\
\cite{Hagen2016GameplayExercise}&Pedal Tanks&\cellcolor{green!30}$>$&Walking Outdoors , self selected pace\\
\cite{Moholdt2017ExergamingTraining}&Pedal Tanks&\cellcolor{green!30}$>$&Walking Outdoors , self selected pace\\
\cite{miranda2017energy}&Rock Band&\cellcolor{red!30}$<$&Treadmill at 30\% VO2 max\\
\cite{Bailey2011EnergyExergaming}&Sportwall&\cellcolor{green!30}$>$&Treadmill at 4.8 km/h\\
\cite{Perusek2014AAdults}&Wii Boxing&\cellcolor{yellow!30}$=$&Heavy Bag Boxing\\
\cite{Sell2011EnergyActivities}&Wii Boxing&\cellcolor{red!30}$<$&Prusik Climbing\\
\cite{Willems2009ComparisonWalking}&Wii Boxing&\cellcolor{yellow!30}$=$&Treadmill at "brisk walk"\\
\cite{Bailey2011EnergyExergaming}&Wii Boxing&\cellcolor{yellow!30}$=$&Treadmill at 4.8 km/h\\
\cite{Sell2011EnergyActivities}&Wii Boxing&\cellcolor{yellow!30}$=$&Treadmill at "brisk walk"\\
\cite{Barkley2009PhysiologicAdults}&Wii Boxing&\cellcolor{green!30}$>$&Treadmill at 4 km/h\\
\cite{Graf2009PlayingChildren}&Wii Boxing&\cellcolor{yellow!30}$=$&Walking at 5.7 km/h\\
\cite{delCorral2014PhysiologicalCf}&Wii Fit&\cellcolor{red!30}$<$&6 Minute Walk Test\\
\cite{Chan2012InteractiveEfficacy}&Wii Fit&\cellcolor{yellow!30}$=$&Arm Ergometer\\
\cite{CebollaIMart2015AlternativeActivos}&Wii Fit&\cellcolor{yellow!30}$=$&Treadmill at 4.2 km/h\\
\cite{Graves2010TheAdults}&Wii Fit&\cellcolor{red!30}$<$&Treadmill at "brisk walk"\\
\cite{Douris2012ComparisonAdults}&Wii Fit&\cellcolor{green!30}$>$&Treadmill at 5.6 km/h\\
\cite{RoopchandMartin2016IsRunning}&Wii Fit&\cellcolor{red!30}$<$&Treadmill Running , self selected pace\\
\cite{Boese2014ESportsErgometertraining}&Wii Fit, Boxing&\cellcolor{red!30}$<$&Stationary Bike at "moderate effort"\\
\cite{Gao2015AExergaming}&Wii Fit, Dance&\cellcolor{green!30}$>$&Physical Education Lesson\\
\cite{Gao2015AExergaming}&Wii Fit, Dance&\cellcolor{yellow!30}$=$&School Recess (Free Play)\\
\cite{Willems2009ComparisonWalking}&Wii Sports&\cellcolor{red!30}$<$&Treadmill at "brisk walk"\\
\cite{White2011EnergyGames}&Wii Sports, Fit&\cellcolor{red!30}$<$&Running\\
\cite{Naugle2014CardiovascularTool}&Wii Sports, Fit&\cellcolor{red!30}$<$&Stationary Bike at "light effort"\\
\cite{Naugle2014CardiovascularTool}&Wii Sports, Fit&\cellcolor{red!30}$<$&Treadmill at "light effort"\\
\cite{White2011EnergyGames}&Wii Sports, Fit&\cellcolor{yellow!30}$=$&Walking, self selected pace\\
\cite{Staiano2011WiiExpenditure}&Wii Tennis&\cellcolor{yellow!30}$=$&Tennis lesson\\
\cite{Bailey2011EnergyExergaming}&Xavix&\cellcolor{green!30}$>$&Treadmill at 4.8 km/h\\
\cite{Monedero2017EnergyExercise}&Your Shape&\cellcolor{red!30}$<$&Treadmill Running, self selected pace\\
\cite{Monedero2017EnergyExercise}&Your Shape&\cellcolor{yellow!30}$=$&Treadmill Running at 55\% VO2 reserve\\

% end paste here

\hline
& \multicolumn{3}{l}{(DDR = Dance Dance Revolution)}

\end{tabular}
}
\end{table}

\begin{table}
\caption{Differences in Enjoyment and Perceived Effort Between Exertion Games and Exercise}
\label{tab:enjoyment}
\centering

\begin{tabularx}{\linewidth}{l @{}p{1.3cm}@{} p{2.5cm} X}

\hline
& \textbf{Game}
 & \textbf{Exercise}
 & \textbf{Enjoyment / RPE / Movement}
 \\
\hline
\cite{Gao2013ChildrenSDance}&DDR&Aerobic Dance&DDR less exercise but higher self efficacy and enjoyment.\\
\cite{Haddock_2009}&Disney Cars&Stationary Bike&Game increased exercise, perceived effort no different\\
\cite{Perron2012ComparisonAdults}&EA Sports&Treadmill at "brisk walk"&Mixed results for exercise but games more enjoyable \\
\cite{Monedero2015InteractiveAdults}&Gamebike&Stationary Bike at 55\% peak power&Game increased exercise, perceived effort no different\\
\cite{eason2016comparison}&Just Dance&Dance Fitness Class&Enjoyment higher in dance class\\
\cite{Jenny2015VirtualPerceptions}&Kinect Climbing&Wall climbing&different lower body movement\\
\cite{CebollaIMart2015AlternativeActivos}&Wii Fit&Treadmill at 4.2 km/h&Same exercise level, game more satisfying\\
\cite{Devereaux2012ComparisonExercise}&Wii Fit&Treadmill at 65\% max hr&Fixed effort level was perceived as lower in games\\
\cite{Devereaux2012ComparisonExercise}&Wii Fit&Stationary Bike at 65\% max hr&Fixed effort level was perceived as lower in games\\
\cite{Boese2014ESportsErgometertraining}&Wii Fit, Boxing&Stationary Bike at "moderate effort"&Games were less exercise but more enjoyable\\
\cite{Naugle2014CardiovascularTool}&Wii Sports, Fit&Treadmill at "light effort"&Games were less exercise but more enjoyable\\
\cite{Naugle2014CardiovascularTool}&Wii Sports, Fit&Stationary Bike at "light effort"&Games were less exercise but more enjoyable\\

\hline
& \multicolumn{3}{l}{(DDR = Dance Dance Revolution)}
\end{tabularx}
\vspace{-.7cm}

\end{table}

Table \ref{tab:gamevsexercise} shows comparisons in 60 studies in our dataset, between games and traditional exercise in terms of whether measured activity levels were greater or lower in the exertion game compared to the exercise. Positive (20), negative (18) and even (19) comparisons between game and exercise are roughly balanced. In particular, compared to walking on a treadmill, or gently riding a stationary bicycle, exertion games in many cases can at least equal the intensity of exercise. 

However, one major criticism of this body of work is exactly the comparison described above; that the vast majority of these studies compare exertion gaming to walking on a treadmill or cycling on a stationary bike, often with no other stimulation. These control conditions arguably flatter exertion games in that they are comparing against the most monotonous and simple forms of exercise, performed at low intensity. In contrast, studies comparing gaming to instructor led aerobic dance \cite{Gao2013ChildrenSDance,eason2016comparison}, climbing exercises \cite{Sell2011EnergyActivities} and Irish dance lessons \cite{Rincker2017TheFitness} all found exertion games produced less physical activity. 

Beyond intensity, Table \ref{tab:enjoyment} lists significant differences observed in relation to enjoyment and/or perceived exertion. We can see from this table that compared to control conditions, games were seen as more enjoyable than the exercise controls, although this was not always in tandem with greater physical activity. There is also some evidence \cite{Haddock_2009,Monedero2015InteractiveAdults,Devereaux2012ComparisonExercise} that players may perceive effort as relatively lower in exertion games compared to non-distracting control conditions. However, again these control conditions were largely uninteresting treadmill or stationary cycling exercises, which limits the strength of these results. Interestingly, in the case of aerobic dance classes versus dance games, a study with children found they enjoyed games more \cite{Gao2013ChildrenSDance}, whereas a study with adults found that they preferred instructor led fitness lessons \cite{eason2016comparison}. 

One key fact that is clear from this research is that games which simulate exercise activities such as Wii Tennis are different to the underlying exercise activities. One interesting study compares the specifics of this difference in relation to rock climbing games \cite{Jenny2015VirtualPerceptions}, showing that while climbing games do not teach the same movement skills, they did however require similar tactical skills to real climbing, so could potentially aid in that element of skill development.

\subsection{How do different features of exertion games alter participant exercise levels and experience?}

In this section, we discuss research which investigates differences in features of exertion games (33 papers), and effects they have on playing. We break these down thematically rather than an exhaustive list of studies as in several cases multiple studies have considered the features described, and also, results are somewhat more nuanced than pure between group comparisons of exertion level or enjoyment described in the previous sections. Table \ref{tab:features} shows these results.

\begin{table}
\caption{Effect of Different Features of Exertion Games}
\label{tab:features}
\centering
\hyphenpenalty=10000
\begin{tabularx}{\linewidth}{@{}>{\raggedright}p{1.5cm}X@{}}

\hline
\textbf{Factor} & \textbf{Effects} \\
\hline

Type of opponent or partner &

{
\begin{tabularx}{\linewidth}[t]{@{}X@{}}
    People play more vigorously against a computer opponent than alone \cite{Mcwha_2009,Feltz2011BuddyGames,Feltz2014CyberExergames,Samendinger2017IntroductoryPartners,Feltz2012TwoPlayerActivity,Staiano2011WiiExpenditure}\\
People play more vigorously with a human opponent than with a computer \cite{Feltz2014CyberExergames,ODonovan2012EnergyModes,Verhoeven2015EnergyEnjoyment,Kooiman2016ExergamingEducation}\\
Except one study, with less exercise when humans cooperated in a shared physical space than single player \cite{Peng2013PlayingExertion}.\\
People perceive less effort for similar exertion in multiplayer modes \cite{Lisn2015CompetitiveAdolescents}, and self reported motivation is higher \cite{Peng2013PlayingExertion}.  

\end{tabularx}}
\\
How other players behave & 
{
\begin{tabularx}{\linewidth}[t]{@{}X@{}}
    One study found that players use more energy when they play cooperatively than competitively, especially when playing with friends. \cite{Peng2012TheGame}.\\
    Whilst another found that local co-op play was lower intensity than remote competitive play \cite{Peng2013PlayingExertion}.\\
    People are more active when their opponent is slightly better than them (rather than worse, or a lot better)\cite{Feltz2012TwoPlayerActivity}\\
    If a virtual exercise \textbf{partner} gives you encouragement in the form of "you" statements, it decreases performance compared to a silent partner.\cite{Max_2016}\\
    If a virtual exercise \textbf{trainer} gives you encouragement in the form of "you" statements as you exercise alone, it increases performance.\cite{Max_2016}
    
\end{tabularx}}
\\
How games are framed
& {
\begin{tabularx}{\linewidth}[t]{@{}X@{}}
    People do more exercise in fitness themed games than entertainment themed games \cite{Chen2014HealthifyingPriming,Monedero2017EnergyExercise,Lyons2012NovelGames}.\\
    But people enjoy entertainment games more \cite{Lyons2012NovelGames}\\
    Players moved more when narrative cutscenes were added to Wii Sports \cite{Lu2016TheStudy}\\
    Players move less if avatars in games are obese \cite{Pea2016IExergames,Pea2014IncreasingAppearance,Li2014WiiExergames}\\
    Players with low body image dissatisfaction preferred to see their image on screen and vice versa.\cite{Song2011PromotingExergame}
    
\end{tabularx}}
\\

Choice of game or play style
& {
\begin{tabularx}{\linewidth}[t]{@{}X@{}}
    People exercise harder when playing a variety of games, compared to a single game \cite{Raynor2016EffectActivity}.\\
    Strong preferences for lower effort games \cite{Bissell2016EffectivenessActivity}.\\
    More exercise in the same game standing than seated \cite{sanders2014physiologic}\\
    People exercise harder when they find games: more engaging, enjoyable, or if the physical actions are `realistic' \cite{Lyons2014EngagementPlay,Lin2015TheGames}

\end{tabularx}}
\\
Adaptiveness and Skill	
& {
\begin{tabularx}{\linewidth}[t]{@{}X@{}}
Games that track heart rates can encourage players to push themselves harder or reach target heart rates \cite{Ketcheson2015DesigningExergames,Sinclair2009ExergameModel}.\\
Games which adapt to player fitness levels increase enjoyment and motivation	\cite{MartinNiedecken2017GoPlanet}\\
	Games that exaggerate player's physical abilities are more enjoyable than those that present realistic depictions	\cite{Kajastila2014EmpoweringGame}\\
\end{tabularx}}
\\

\hline

\end{tabularx}
\vspace{-.7cm} 


\end{table}

These articles demonstrate that a range of factors affect how people exercised in games. One key factor is multiplayer modes - with multiple studies showing that people exercised more vigorously when against an opponent, and yet more vigorously when against a human opponent than against a computer. One interesting study found in contrast that humans playing kinect in a limited shared space exercised less hard than in single player modes, presumably due to space constraints \cite{Peng2013PlayingExertion}. Multiplayer modes also engendered reduced perceived effort and greater motivation in participants. 

The type of multiplayer mode and partner behaviour also affects exertion. On competitive vs cooperative play, results are currently unclear, with one study \cite{Peng2012TheGame} reporting that participants use more energy playing  cooperatively than competitively, whilst another by the same authors showed less effort in local cooperative play versus remote competitive play \cite{Peng2013PlayingExertion}. Studies with virtual partners have also demonstrated the Köhler effect \cite{witte1989kohler}, a well known effect in which performance is highest when a virtual exercise partner is manipulated to be just slightly better than the player, rather than worse or a lot better \cite{Feltz2012TwoPlayerActivity}. Similarly, how a virtual character talks to the player, and the role of that character can also affect exercise performance, with one study showing that if exercise \textit{partners}, who are also shown to be exercising, give encouragement such as "You can do it", this actually decreases performance, whereas if a non-exercising exercise \textit{trainer} encourages players similarly, performance is increased \cite{Max_2016}.

Another feature that can affect games is how the games are framed - this can be in terms of deep differences in theme of games, where multiple studies have shown players to do more exercise in games which are themed as exercise games than entertainment, but to enjoy the exercise games less. However effects have also been demonstrated with even simple cosmetic manipulations such as making avatars obese (which reduced exercise performance), or adding narrative cutscenes to a game (which increased exercise performance).

Studies comparing multiple games, or choices of games found that variety of games encourages more exercise, although when given choices, players preferred games that are less effort to play. People were found to exercise harder with games they found more engaging or enjoyable, but also in games that were felt to be more realistic. One way in which such enjoyment and engagement could be found was in games that adapted well to player fitness levels or heart rate; another study showed that providing exaggerated representations of player's physical abilities also increased enjoyment.

These articles show that a wide range of exertion game features affect the heavily interlinked factors of how people exercise and how enjoyable or motivating they find games. Many factors are not even directly about gameplay, such as visual framing, the presented purpose or theme of the game, the variety of choice of games, and who or what a person is playing the game with. 

\subsection{How do different user groups experience exertion games differently?}

\begin{table}
\caption{Ages of Study Participants}
\label{tab:demographics}
\centering
\begin{tabular}{lc}
\hline
Age Group & Number of Studies\\
\hline
Children &77\\
College age adults &71\\
Adults (other) & 34\\
Elderly&12\\
\hline
\end{tabular}
\vspace{-.7cm}
\end{table}

One clear limitation of exertion gaming research relating to exercise levels is that the vast majority of work has been done on children or college age adults. Table \ref{tab:demographics} shows the ages of study participants in our dataset (n.b. this is only articles relating to short term exercise - a large amount of work has been done in relation to rehab and balance in the elderly [REF]). Younger people are well known to exercise more \cite{bhf_2015},  there is limited evidence as to the effects of exertion games particularly on middle aged adults, who are likely to do less exercise in general.  

Whilst we can say little about adults of different ages, we did find a range of research which compares how paired groups of users grouped by personal characteristics play exertion games (19 papers). This work is summarized in Table \ref{tab:groups}. 

\begin{table}
\caption{How Different Player Groups Experience Exertion Games}
\label{tab:groups}
\centering
\hyphenpenalty=10000

\begin{tabularx}{\linewidth}{@{}l@{}llXl@{}}
\hline
& \textbf{Game} & \textbf{Group 1} & \textbf{Comparison with} & \textbf{Group 2} \\
\hline 

% paste here
\multicolumn{5}{c}{\textbf{Gender and Age}}\\
\cite{Clevenger2015EnergyKinect}&Kinect&Male&more energy used&Female\\
\cite{McNarry2016InvestigatingChildren}&K. Adventures&Male&more energy used&Female\\
\cite{Lam2011PlayAlternatives}&Xavix Bowling&Male&more active than&Female\\
\cite{Lin2015JustOutcomes}&Just Dance&Male&same energy used&Female\\
\cite{Clevenger2015EnergyKinect}&Kinect&Teenagers&use more energy&Children\\
\multicolumn{5}{c}{\textbf{Weight Status}}\\
\cite{ODonovan2014TheWeight}&Wii Fit&Obese&less energy used&Normal\\
\cite{Pope2016EffectsRevolution}&DDR&Overweight&less active than&Normal\\
\cite{Clevenger2015EnergyKinect}&Kinect&Overweight&more energy used&Normal\\
\cite{Unnithan2006EvaluationAdolescents}&DDR&Overweight&same VO2 as&Normal\\
\cite{Chaput2016LeanOnes}&K. Boxing&Overweight&less active than&Normal\\
\cite{Lau2015EvaluatingChildren}&Various&Overweight&same energy used&Normal\\
\cite{Bailey2011EnergyExergaming}&Various&Overweight&same exercise, but enjoyed more&Normal\\
\multicolumn{5}{c}{\textbf{Health Status}}\\
\cite{Robert2013ExerciseConsole}&Wii Fit&With CP&same energy used&Healthy\\
\cite{Kafri2014EnergyPoststroke}&Various&Post-stroke&less active than&Healthy\\
\cite{Neil2013SonyRehabilitation}&Various&Post-stroke&less active than&Healthy\\
\multicolumn{5}{c}{\textbf{Player Skill}}\\
\cite{Sell2008EnergyExperience}&DDR&Skilled&more active than&Non-skilled\\
\cite{Soltani2017PhysiologicalExperience}&K. Swimming&Skilled&played less time&Non-skilled\\
\cite{Berkovsky2012PhysicalReward}&Neverball&Skilled&did less execise&Non-skilled\\
\cite{Soltani2017PhysiologicalExperience}&K. Swimming&Swimmers&equally active&Non swimmers\\
\cite{Soltani_2016}&K. Swimming&Swimmers&equally good&Non-swimmers\\

% finish paste here
\hline
& \multicolumn{4}{l}{(K. = Kinect, DDR = Dance Dance Revolution)
}
\end{tabularx}
\vspace{-.7cm}

\end{table}

Multiple studies \cite{Clevenger2015EnergyKinect,McNarry2016InvestigatingChildren,Lam2011PlayAlternatives} (although not all \cite{Lin2015TheGames}) found that male players exercised harder in exertion games. This fits with known disparities in levels of exercise particularly amongst teenage girls (e.g. \cite{te2007patterns}). We note that studies only with adolescent girls have achieved positive results \cite{Fawkner2010AdolescentGaming}.

Weight status also showed effects on how people played exertion games, although as can be seen in Table \ref{tab:groups}, results of weight status are highly varied, with overweight groups using more, the same, or less energy (normalised for weight), than normal weight participants. Whilst this research has yet to provide clear answers, it is clear that to be effective with a range of participants,  game design and testing must take into account the need of those with differing weight status.

Health status also appears to have mixed results on gameplay, but shows some evidence that health conditions such as stroke may reduce intensity of play.

Finally, research has considered the effect of player skill on gaming intensity. This seems highly game specific - for example in Dance Dance Revolution, high skilled players can dance faster and were more active than other players; whereas in Neverball \cite{Berkovsky2012PhysicalReward} and Kinect Michael Phelps Swimming \cite{Soltani2017PhysiologicalExperience}, players who were more skilled at the game were able to do less exercise. Another key point of this research is that skill in exertion games often isn't the same as the actual exercise that is being simulated, with for example swimming skill not correlating to improved performance in a swimming exertion game \cite{Soltani2017PhysiologicalExperience}, this is posited to be due to the motions in the game being different to real swimming, similar to the climbing games discussed above \cite{Jenny2015VirtualPerceptions}.

\section{Conclusion and Key Lessons}

The sometimes unspoken hypothesis of exertion games is simple and appears intuitive - if we make games that encourage people to move, they will do exercise, and reap the same health benefits that regular traditional exercise is known to provide, and help people meet public health recommendations for exercise. This paper demonstrates that the reality is more complicated; clearly exertion game play can be exercise, and can encourage exercise to levels that meet public health recommendations. But even in lab settings, many games did not encourage even moderate levels of exercise. It is clearly far more complicated than assuming that if one builds a game which encourages people to move, they will move when they play it, and that movement will reach recommended intensity levels. We suggest that this body of research leads to three key lessons for researchers in the area:

\subsection{Context is important}
Doing exercise is clearly a complex social behaviour which involves far more than the moment of exercise itself. However, what is evaluated in the short-term exercise here is purely the moment of gaming. We can see from the massive variation in results for games which were studied multiple times such as Wii Boxing and Dance Dance Revolution that it is hard to say that any specific game is exercise. Whilst there are clear patterns here, as observed in previous research \cite{Peng_2011} that lower body games such as dance games lead to higher typical exertion levels, even within such games we see ranges from less than 3 METs exertion to greater than 6. Clearly the level of exercise may be affected by many factors such as participant type and skill level, difficulty level choices, and experimental conditions such as instructions given, time structure of experiments, what other activities were done in the experiment, what display was used etc. As such, given the fact that in real-world settings, people typically have control over their game systems, and a choice as to whether to play the game and how hard to play, it is hard to argue that this work demonstrates that an exertion game is superior to for example a ball, or other piece of participant choice exercise equipment, which given a well chosen context and participants who choose to exercise are well known to lead to high levels of exertion. 

One key target context of exertion games is schools; in this context, where a level of coercion exists, and settings are more controlled, there is some promise, with one study \cite{Gao2015AExergaming} showing increased exertion levels in exertion gaming versus standard physical education lessons. However the children in this study also achieved similarly increased levels of exertion in free play at recess, so again it shows little evidence of advance over simply letting children free to run outside.

\subsection{Exercise is more than exertion}

A fundamental point which is missing in measurement of exertion games by the quantity of exertion performed is the wider purpose of exercise. For example whilst school physical education may use less energy than some exertion games \cite{Gao2015AExergaming}, that may have been because those lessons were also teaching children valuable skills which they may take forward into a lifetime of doing exercise, a benefit which it is unclear exertion games have. Similarly, whilst most work focuses on aerobic exercise, and many papers reference the need to fulfill the WHO recommendation of 150 minutes moderate exercise per week, few discuss the accompanying recommendation for strength training. 
In addition to this, specifically in interventions targeted at adults, there is increasing evidence that reducing prolonged sedentary behaviour may have strong positive effects independent of quantity of moderate to vigorous exercise (e.g.\cite{healy2008objectively}). As such, regular getting up and moving may be more important than intensity - meaning that studying ongoing regular engagement will be more important in such cases. We caution however that in children a previous review showed that negative effects of sedentary behaviour were not found in when exercise levels were controlled for \cite{cliff2016objectively}.

\subsection{We need to consider realistic alternatives to exertion games}

When designing comparative studies of exertion games against exercise, we suggest that rather than compare against trivial and monotonous (but easily arranged) comparison conditions such as treadmill walking or gentle exercise biking, researchers should try and explore realistic comparisons against more entertaining sporting activities. This is particularly relevant when our work may be used to promote such games as alternatives to existing exercise activities - as the data shows, only 10\% of studies reported vigorous activity levels from exertion games, showing that as previously discussed by Canadian child health specialists \cite{Chaput2013ActiveYouth}, such games are not yet equivalent to more intense sports. We can see this in the few studies in our dataset which compare exertion games against non-monotonous forms of exercise \cite{Gao2013ChildrenSDance,Rincker2017TheFitness,Sell2011EnergyActivities} which showed increased activity levels in traditional exercises versus exertion games.

Further to this, the fact that context is so important to exercise impacts of exertion games suggests that for them to have health benefits, they will need to be part of a structured intervention. Such structures are likely to have added costs, over and above hardware and software for games themselves. As such, when considering the potential benefit of exertion games it is important to consider the total cost of providing the intervention and game in comparison to the costs of other interventions such as instructor led exercise which may lead to greater exercise and social benefits.
