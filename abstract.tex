Background: Exertion games are computer games that require full body exercise to play. These games have been widely presented as potential public health interventions, to encourage otherwise sedentary populations to take exercise at levels recommended by health professionals. 

Objectives: For exertion games to bring about exercise-related health benefits, two things must be true. Firstly, playing the games must encourage exercise at levels sufficient to engender such health benefits. Secondly, players must continue to play those games over an extended period of time. We address the first question in this article.

Methods: We carried out a scoping review of empirical research that has examined whether exertion game play engenders exercise. We searched Google Scholar, Scopus and PubMed.

Results: From 3171 search records, we found 726 exertion game studies, of which 243 addressed short term exercise. While some studies found exertion games to engender moderate levels of exertion, players of many exertion games fail to meet recommended activity levels. Few games encouraged vigorous levels seen in sporting activities. A wide spread in results for individual games suggests that exertion game play levels are highly dependent on non-game contextual factors. There is some evidence that games can make exercise more enjoyable or reduce perceived exertion, but many studies have the methodological problem of considering extremely boring control conditions. The vast majority of studies focus on college-age students.

Conclusions: As exertion game researchers, we need to design our studies taking into account three factors: 1) exertion games are highly modulated by context of play, 2) exercise is more complex than just quantity of exertion, 3) we need to consider what exertion games are an alternative to and use realistic controls in comparative studies.
