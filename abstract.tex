Background: Exertion games are video games that require exercise. They are widely presented as health interventions, to encourage sedentary populations to take exercise at levels recommended by health professionals. 

Objectives: We consider whether games encourage exercise at levels sufficient to engender exercise-related health benefits, and in what conditions that occurs. 

Methods: We performed a scoping review of empirical research that examines whether exertion game play engenders exercise, searching Google Scholar, Scopus and PubMed.

Results: From 3171 search records, we found 243 studies of  short-term exercise in games. While some observed moderate levels of exertion, players of many games fail to meet recommended levels. Few games encouraged vigorous levels seen in sports. Variation in results for games across different studies suggests that exertion motivation is highly dependent on non-game contextual factors. There is evidence games make exercise more enjoyable or reduce perceived exertion, but many studies suffer the methodological problem of comparison with boring control conditions.

Conclusions: Exergames have only been found comparable to exercise such as walking, jogging and dancing under very specific circumstances. To improve evidence for games as exercise interventions, we must improve study designs and focus on understanding better the circumstances likely to bring about genuine exergame exercise. 
